\vspace*{\fill}
\begin{abstract}
	The Markov Chain Monte Carlo methods (abbrv. MCMC) are a family of algorithms used for approximating sampling from a given probability distribution. They prove very effective when the state space is large. This fact can be used to solve many hard deterministic problems -- one of them being \textit{traveling salesmen problem}. It will be used in this thesis to test a new approach of \textit{locally-informed proposals} as a modification of well known \textit{Metropolis-Hastings} algorithm. In this thesis we will present the implementation of modified algorithm, experiments based on it, results and a comparison with previous MCMC methods.
	
	\rule{0.8\textwidth}{0.4pt}
	
	Metody próbkowania Monte Carlo łańcuchami Markowa są rodziną algorytmów używanych do przybliżania próbkowania z danego rozkładu prawdopodobieństwa. Okazują się efektywne zwłaszcza gdy przestrzeń stanów jest wielka. Ten fakt może być wykorzystany przy rozwiązywaniu wielu deterministycznych problemów -- jednym z nich jest \textit{problem komiwojażera}. Zostanie on użyty w tej pracy do przetestowania nowego podejścia wykorzystującego \textit{lokalnie-poinformowane rozkłady generujące kandydatów}, jako modyfikacji dobrze znanego algorytmu \textit{Metropolisa-Hastinigsa}. W tej pracy zaprezentujemy implementacje zmodyfikowanego algorytmu, eksperymentów bazujących na nim, wyników oraz porównania z poprzednimi metodami próbkowania Monte Carlo.
\end{abstract}
\vspace*{\fill}