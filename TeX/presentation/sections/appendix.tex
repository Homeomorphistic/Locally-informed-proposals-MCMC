\begin{frame}
	\frametitle{Markov chain}
	\begin{definition}[State space]
		 A state space of a Markov chain is a countable set $S$.
	\end{definition}
	\begin{definition}[Index set]
		An index set of a Markov chain is a countable set $T$.
	\end{definition}
\end{frame}

\begin{frame}
	\frametitle{Markov chain}
	\begin{definition}
		A Markov chain is a sequence of random variables $\left\{X_k\right\}_{k \in T}$ defined on a common probability space $\left( \Omega, \mathcal{F}, P \right)$, that take values in $S$, such that it satisfies Markov property:
		\begin{align*}
			&P(X_{k+m} = j | X_k = i, X_{l_{p-1}} = i_{l_{p-1}}, \ldots, X_{l_1} = i_1) = \\= &P(X_{k+m} = j | X_k = i),
		\end{align*}
		for all indices $l_1< \ldots < l_{p-1} < k < k+m, \; 1 \leq p \leq k$, all states $j, i, i_{p-1}, i_{p-2}, \ldots, i_0 \in S$ and $m \geq 1$.
	\end{definition}
\end{frame}

\begin{frame}
	\frametitle{Properties}
	\begin{definition}[Irreducibility]
		A Markov chain is called irreducible if and only if for every pair of states $i$ and $j$ there exists a positive probability of transition between them.
	\end{definition}
	\begin{definition}[Periodicity]
		Let $d_i$ be a greatest common divisor of those $k$ such that $\PP_{i,i}(k)>0$. If $d_i > 1$ then state $i$ is periodic. If $d_i = 1$ then state $i$ is aperiodic.
	\end{definition}
\end{frame}

\begin{frame}
	\frametitle{Properties}
	\begin{definition}[Stationarity]
		A probability distribution $\bfpi = (\pi_1, \ldots, \pi_N)$ is called stationary if it satisfies
		\begin{equation*}
		\pi_j = \sum_{i \in S} \pi_i p_{ij},
		\end{equation*}
		or equivalently in vector form:
		\begin{equation*}
		\bfpi = \bfpi \PP.
		\end{equation*}
		This equation is often described as the balance equation.
	\end{definition}
	\begin{definition}[Ergodicity]
		A Markov chain is ergodic when it is irreducible and aperiodic.
	\end{definition}
\end{frame}

\begin{frame}
	\frametitle{Traveling salesman problem}
	\begin{definition}
		A undirected graph $G$ is a pair $(V, E)$, where $V$ is a set of vertices and $E$ is a set of edges, which is a subset of all unordered pairs of vertices.
	\end{definition}
	\begin{definition}
		A tour is a Hamilitonian cycle and we identify it with a permutation of vertices.
	\end{definition}
\end{frame}

\begin{frame}
	\frametitle{Random candidates (RN)}
	\begin{algorithm}[H]
	\tiny
	\caption{Random neighbours algorithm}\label{alg:rnd_neigh}
	\begin{algorithmic}[1]
		\State{Choose a tour $\sigma \in S_n$.}
		\State{$X_0 \gets \sigma$}
		\State{Compute weight $w_\sigma$.}
		\For{$i = 0,1, \ldots$}
		\State{Sample $k, l \sim Unif\left\{1, 2, \ldots, n\right\}$ without replacement.}
		\State{Sample $U \sim Unif(0,1)$.}
		\State{$w_{\sigma'} \gets w_\sigma - (w_{\sigma(k-1) + \sigma(k)} + w_{\sigma(k) + \sigma(k+1)}, w_{\sigma(l-1) + \sigma(l)} + w_{\sigma(l), \sigma(l+1)}) $}
		\State{$w_{\sigma'} \gets w_\sigma' + (w_{\sigma(k-1) + \sigma(l)} + w_{\sigma(l) + \sigma(k+1)} + w_{\sigma(l-1) + \sigma(k)} + w_{\sigma(k) + \sigma(l+1)})$}
		\If{$\log(U) \leq \min \left(0, -(w_{\sigma'} - w_\sigma) \right)$}
		\State $X_{i+1} \gets X_i$
		\State $X_{i+1}(k), X_{i+1}(l) \gets X_{i+1}(l), X_{i+1}(k)$
		\State{$w_{\sigma} \gets w_{\sigma'}$}
		\Else
		\State $X_{i+1} \gets X_i$
		\EndIf
		\EndFor
	\end{algorithmic}
\end{algorithm}
\end{frame}

\begin{frame}
	\frametitle{Locally-informed proposals (LIP)}
	\begin{algorithm}[!htb]
	\caption{Locally-informed proposals algorithm}\label{alg:loc_neigh}
	\begin{algorithmic}[1]
		\State{Choose a tour $\sigma \in S_n$.}
		\State{$X_0 \gets \sigma$}
		\State{Compute weight $w_\sigma$.}
		\State{Compute all neighbour weight differences $\bfd_\sigma$.}
		\State{$s(\bfd_\sigma) = \mathrm{softmax}(\bfd_\sigma)$}
		
		\For{$i = 1,2, \ldots$}
			\State{Sample $\sigma' \sim s(\bfd_\sigma)$.}
			\State{Find $k, l$ connected with swapping.}
			
			\State{$w_{\sigma'} \gets w_\sigma + \bfd_\sigma[(k,l)]$}
			\State{$\bfd_{\sigma'} \gets \bfd_\sigma$}
			
			\State{$\bfd_{\sigma'} \gets$ \textit{update\_differences}$(\bfd_{\sigma'})$}
			
			\State{$s(\bfd_{\sigma'}) = \mathrm{softmax}(\bfd_{\sigma'})$}
			\State{Sample $U \sim Unif(0,1)$.}
			\If{$\log(U) \leq \min \left(0, -(w_{\sigma'} - w_\sigma) + \log(s(\bfd_{\sigma'})[(k,l)]) - \log(s(\bfd_\sigma)[(k,l)]) \right)$}
				\State $X_{i+1}(k), X_{i+1}(l) \gets X_{i+1}(l), X_{i+1}(k)$
				\State{$w_{\sigma} \gets w_{\sigma'}$}
				\State $\bfd_{\sigma} \gets \bfd_{\sigma'}$
			\Else
				\State $X_{i+1} \gets X_i$
			\EndIf
		\EndFor
	\end{algorithmic}
\end{algorithm}
		
The main algorithm uses a sub-algorithm \ref{alg:update_diff} which updates weights in swaps that we need to consider separately. It uses function \textit{get\_difference}, which is just getting appropriate weights to remove and add. 
\begin{algorithm}[!htb]
	\caption{update\_differences}\label{alg:update_diff}
	\begin{algorithmic}[1]
		\Require $\bfd_{\sigma'}$
		\Ensure $\bfd_{\sigma'}$
		\For{$r = 1, 2, \ldots, n$}
			\For{$s = r+1, \ldots, n$}
				\If{$r \; \mathrm{or} \; s \; \mathrm{in} \left\{ (k-1), k, (k+1), (l-1), l, (l+1) \right\}$}
					\State{$\bfd_{\sigma'}[(m,n)] \gets$ \textit{get\_difference}($\bfd_{\sigma'}[(r,s)]$)}
				\EndIf
			\EndFor
		\EndFor
	\end{algorithmic}
\end{algorithm}

\begin{algorithm}[!htb]
	\caption{update\_differences\_alt}\label{alg:update_diff_alt}
	\begin{algorithmic}[1]
		\Require $\bfd_{\sigma'}$
		\Ensure $\bfd_{\sigma'}$
		\For{$r = k-1, k, k+1, l-1, l, l+1$}
			\For{$s = r+1, \ldots, n$}
				\State{$\bfd_{\sigma'}[(m,n)] \gets$ \textit{get\_difference}($\bfd_{\sigma'}[(r,s)]$)}
			\EndFor
		\EndFor
		
		\For{$r = k+2, k+3, \ldots, l-2$}
			\For{$s = l-1, l, l+1$}
				\State{$\bfd_{\sigma'}[(m,n)] \gets$ \textit{get\_difference}($\bfd_{\sigma'}[(r,s)]$)}
			\EndFor
		\EndFor
	\end{algorithmic}
\end{algorithm}
Alternate algorithm explanation: ??
\end{frame}

\begin{frame}
	\frametitle{Locally-informed proposals (LIP)}
	\begin{algorithm}[H]
	\tiny
	\caption{update\_differences}
	\begin{algorithmic}[1]
		\Require $\bfd_{\sigma'}$
		\Ensure $\bfd_{\sigma'}$
		\For{$r = k-1, k, k+1, l-1, l, l+1$}
		\For{$s = r+1, \ldots, n$}
		\State{$\bfd_{\sigma'}[(r,s)] \gets$ \textit{get\_difference}($\bfd_{\sigma'}[(r,s)]$)}
		\EndFor
		\EndFor
		
		\For{$s = k-1, k, k+1, l-1, l, l+1$}
		\For{$r = 1, 2, \ldots l$}
		\State{$\bfd_{\sigma'}[(r,s)] \gets$ \textit{get\_difference}($\bfd_{\sigma'}[(r,s)]$)}
		\EndFor
		\EndFor
	\end{algorithmic}
\end{algorithm}
\end{frame}

\begin{frame}
	\frametitle{Locally-informed proposals (LIP)}
	Choose tour $\sigma$ and its neighbour $\sigma'$ that is connected with swapping indices $k$ and $l$.
	\begin{align*}
	\sigma = (&\ldots, \sigma(k-1), \sigma(k), \sigma(k+1), \ldots, \\
	&\ldots, \sigma(l-1), \sigma(l), \sigma(l+1), \ldots) \\
	\sigma' = (&\ldots, \sigma(k-1), \sigma(l), \sigma(k+1), \ldots, \\
	&\ldots, \sigma(l-1), \sigma(k), \sigma(l+1), \ldots)
	\end{align*}
\end{frame}

\begin{frame}
	\frametitle{Locally-informed proposals (LIP)}
	The neighbours $\sigma_{r,s}$, $\sigma'_{r,s}$ of $\sigma$ and $\sigma'$ respectively look like: 
	\begin{align*}
	\sigma_{r,s} = &(\ldots, \sigma(r-1), \sigma(s), \sigma(r+1), \ldots, \\
	&\ldots, \sigma(s-1), \sigma(r),\sigma(s+1), \ldots, \\
	&\ldots, \sigma(k-1), \sigma(k), \sigma(k+1), \ldots, \\
	& \ldots, \sigma(l-1), \sigma(l), \sigma(l+1), \ldots) \\
	\sigma'_{r,s} = &(\ldots, \sigma(r-1), \sigma(s), \sigma(r+1), \ldots \\
	& \ldots, \sigma(s-1), \sigma(r), \sigma(s+1), \ldots, \\
	&\ldots \sigma(k-1), \sigma(l), \sigma(k+1), \ldots, \\
	& \ldots, \sigma(l-1), \sigma(k), \sigma(l+1), \ldots)
	\end{align*}
\end{frame}

\begin{frame}
	\frametitle{Example}
	Let us set $r=3$ and $s=9$, which means that we are looking for weight differences of neighbours obtained by swapping $3$ and $9$. So the neighbours have form:
	\begin{align*}
	\sigma_{3,9} &= (1, 2, 9, 4, 5, 6, 7, 8, 3, \ldots) \\
	\sigma'_{3,9} &= (1, 7, 9, 4, 5, 6, 2, 8, 3, \ldots).
	\end{align*}
	\begin{equation*}
	w_{\sigma_{3,9}} - w_{\sigma'_{3,9}} = (w_{1,2} + w_{2,9} + w_{6,7} + w_{7,8}) - (w_{1,7} + w_{7,9} + w_{6,2} + w_{2,8}).
	\end{equation*}
	We can generalize that equation for any $p \notin \left\{ 1,2,3,6,7,8 \right\}$:
	\begin{equation*}
	w_{\sigma_{3,p}} - w_{\sigma'_{3,p}} = f_3(p) = C + w_{2,p} - w_{7,p}.
	\end{equation*}
\end{frame}