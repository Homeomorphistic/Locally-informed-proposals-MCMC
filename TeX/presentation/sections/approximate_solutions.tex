\begin{frame}
	\frametitle{Softmax}
	\begin{definition}
		For a given vector $\x = (x_1, x_2, \ldots, x_d)^T \in \R^d$ a softmax function $s: \R^d \rightarrow \left[0, 1\right]^d$ is defined as
		\begin{align*}
		&s(\x)_i = \frac{e^{x_i}}{\sum_{j=1}^{d} e^{x_j}}, \\
		&s(\x) = \left( s(\x)_1, s(\x)_2, \ldots, s(\x)_d \right).
		\end{align*}
	\end{definition}
	\begin{equation*}
		\sigma_{\min} = \underset{\sigma \in S_n}{\argmin} \left( w_{\sigma} \right) = \underset{\sigma \in S_n}{\argmax} \frac{e^{-w_{\sigma}}}{\sum_{\sigma' \in S_n} e^{-w_{\sigma'}}},
	\end{equation*}
\end{frame}

\begin{frame}
	\frametitle{Metropolis-Hastings algorithm}
	\begin{equation*}
		\frac{\pi_j \bfQ_{j,i}}{\pi_i \bfQ_{i,j}} = e^{-(w_j-w_i)} \cdot \frac{\bfQ_{j,i}}{\bfQ_{i,j}}.
	\end{equation*}
	\begin{equation*}
		\log \left( \frac{\pi_j \bfQ_{j,i}}{\pi_i \bfQ_{i,j}} \right) = -(w_j-w_i) + \log(\bfQ_{j,i}) - \log(\bfQ_{i,j}).
	\end{equation*}
\end{frame}

\begin{frame}
	\frametitle{Candidates}
	\begin{definition}
		A neighbour $\sigma'$ of a permutation $\sigma$ is a permutation, that for some $k, l$ it satisfies $\sigma'(k) = \sigma(l)$, $\sigma'(l) = \sigma(k)$ and $\sigma'(i) = \sigma(i)$ for the rest of indices.
	\end{definition}
	These neighbours are the original tour with two swaped indices. This let  us consider a smaller space -- there are $\binom{n}{2} = \frac{n(n-1)}{2} \approx n^2$ neighbours if the number of vertices is $n$.
\end{frame}

\begin{frame}
	\frametitle{Random candidates (RN)}
	Sample neighbours uniformly. It is equivalent to choosing random indices to swap
\end{frame}

\begin{frame}
	\frametitle{Random candidates (RN)}
	When given a tour $\sigma$ and its neighbour $\sigma'$ they differ only on those edges where swap is happening, let us say $k,l$. So for this situation we have tours:
	\begin{align*}
		\sigma = (&\ldots, \sigma(k-1), \sigma(k), \sigma(k+1), \ldots, \\
		&\ldots, \sigma(l-1), \sigma(l), \sigma(l+1), \ldots) \\
		\sigma' = (&\ldots, \sigma(k-1), \sigma(l), \sigma(k+1), \ldots, \\
		&\ldots, \sigma(l-1), \sigma(k), \sigma(l+1), \ldots)
	\end{align*}
	We need to remove weights $w_{\sigma(k-1), \sigma(k)}$, $w_{\sigma(k), \sigma(k+1)}$, $w_{\sigma(l-1), \sigma(l)}$, $w_{\sigma(l), \sigma(l+1)}$ and add $w_{\sigma(k-1), \sigma(l)}, w_{\sigma(l), \sigma(k+1)}, w_{\sigma(l-1), \sigma(k)}, w_{\sigma(k), \sigma(l+1)}$.
\end{frame}

\begin{frame}
	\frametitle{Locally-informed proposals (LIP)}
	The idea is to balance the increase in the probability of neighbour with decrease of reverse probability, such that it will be easy to compute.
	\begin{equation*}
		\bfQ_{i,j} \propto e^\frac{-(w_{j} - w_i)}{\tau}.
	\end{equation*}
	The distribution is chosen in such a way, so that we can easily group up the terms in acceptance criterion:
	\begin{equation*}
		\frac{\pi_j \bfQ_{j,i}}{\pi_i \bfQ_{i,j}} = e^{-(w_j-w_i)} \cdot \frac{e^\frac{-(w_{i} - w_j)}{\tau}}{e^\frac{-(w_{j} - w_i)}{\tau}} \cdot \frac{C_j}{C_i} = e^{\left( -(w_j-w_i) \left(1 - \frac{2}{\tau}\right) \right)} \cdot \frac{C_j}{C_i},
	\end{equation*}
\end{frame}



\begin{frame}
	\frametitle{Example}
	Let us set $k=2$ and $l=7$, then the set of indices to consider is $\left\{ 1,2,3,6,7,8 \right\}$ and the permutations:
	\begin{align*}
	\sigma &= (1, \bm{2}, 3, 4, 5, 6, \bm{7}, 8, 9, \ldots) \\
	\sigma' &= (1, \bm{7}, 3, 4, 5, 6, \bm{2}, 8, 9, \ldots).
	\end{align*}
\end{frame}

\begin{frame}
	\frametitle{Example}
	% Please add the following required packages to your document preamble:
% \usepackage[table,xcdraw]{xcolor}
% If you use beamer only pass "xcolor=table" option, i.e. \documentclass[xcolor=table]{beamer}
\begin{table}[!htb]
	\centering
	\begin{tabular}{lllllllllll}
		& 2                             & 3                             & 4                            & 5                             & 6                             & 7                             & 8                             & 9                             & 10                                &  \\
		1 & \cellcolor[HTML]{9B9B9B}(1,2) & \cellcolor[HTML]{9B9B9B}(1,3) & \cellcolor[HTML]{C0C0C0}(1,4) & \cellcolor[HTML]{C0C0C0}(1,5) & \cellcolor[HTML]{9B9B9B}(1,6) & \cellcolor[HTML]{9B9B9B}(1,7) & \cellcolor[HTML]{9B9B9B}(1,8) & \cellcolor[HTML]{C0C0C0}(1,9) & \cellcolor[HTML]{C0C0C0}$\cdots$ &  \\
		2 &                               & \cellcolor[HTML]{9B9B9B}(2,3) & \cellcolor[HTML]{C0C0C0}(2,4) & \cellcolor[HTML]{C0C0C0}(2,5) & \cellcolor[HTML]{9B9B9B}(2,6) & \cellcolor[HTML]{9B9B9B}(2,7) & \cellcolor[HTML]{9B9B9B}(2,8) & \cellcolor[HTML]{C0C0C0}(2,9) & \cellcolor[HTML]{C0C0C0}$\cdots$ &  \\
		3 &                               &                               & \cellcolor[HTML]{C0C0C0}(3,4) & \cellcolor[HTML]{C0C0C0}(3,5) & \cellcolor[HTML]{9B9B9B}(3,6) & \cellcolor[HTML]{9B9B9B}(3,7) & \cellcolor[HTML]{9B9B9B}(3,8) & \cellcolor[HTML]{C0C0C0}(3,9) & \cellcolor[HTML]{C0C0C0}$\cdots$ &  \\
		4 &                               &                               &                               & (4,5)                         & \cellcolor[HTML]{C0C0C0}(4,6) & \cellcolor[HTML]{C0C0C0}(4,7) & \cellcolor[HTML]{C0C0C0}(4,8) & (4,9)                         & $\cdots$                         &  \\
		5 &                               &                               &                               &                               & \cellcolor[HTML]{C0C0C0}(5,6) & \cellcolor[HTML]{C0C0C0}(5,7) & \cellcolor[HTML]{C0C0C0}(5,8) & (5,9)                         & $\cdots$                         &  \\
		6 &                               &                               &                               &                               &                               & \cellcolor[HTML]{9B9B9B}(6,7) & \cellcolor[HTML]{9B9B9B}(6,8) & \cellcolor[HTML]{C0C0C0}(6,9) & \cellcolor[HTML]{C0C0C0}$\cdots$ &  \\
		7 &                               &                               &                               &                               &                               &                               & \cellcolor[HTML]{9B9B9B}(7,8) & \cellcolor[HTML]{C0C0C0}(7,9) & \cellcolor[HTML]{C0C0C0}$\cdots$ &  \\
		8 &                               &                               &                               &                               &                               &                               &                               & \cellcolor[HTML]{C0C0C0}(8,9) & \cellcolor[HTML]{C0C0C0}$\cdots$ &  \\
		9 &                               &                               &                               &                               &                               &                               &                               &                               & $\ddots$                         &  \\
		&                               &                               &                               &                               &                               &                               &                               &                               &                                  & 
	\end{tabular}
	\caption{The Table representing neighbours.}
	\label{tab:neighbours}
\end{table}
\end{frame}

\begin{frame}
	\frametitle{Simulated annealing}
	The idea is to describe a probability of state using a cooling parameter $t_k$ such that it reminds the cooling of a metal and may change with each step:
	\begin{equation*}
	\pi_i = \frac{e^{\frac{-E_i}{t_k}}}{C},
	\end{equation*}
	where $C$ is normalizing constant.The quotient of probabilities then is:
	\begin{equation*}
	\frac{\pi_j}{\pi_i} = e^{\frac{E_i - E_j}{t_k}}
	\end{equation*}
\end{frame}