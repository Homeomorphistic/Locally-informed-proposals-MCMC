\begin{frame}
	\frametitle{Softmax}
	\begin{definition}
		For a given vector $\x = (x_1, x_2, \ldots, x_d)^T \in \R^d$ a softmax function $s: \R^d \rightarrow \left[0, 1\right]^d$ is defined as
		\begin{align*}
		&s(\x)_i = \frac{e^{x_i}}{\sum_{j=1}^{d} e^{x_j}}, \\
		&s(\x) = \left( s(\x)_1, s(\x)_2, \ldots, s(\x)_d \right).
		\end{align*}
	\end{definition}
	\begin{equation*}
		\sigma_{\min} = \underset{\sigma \in S_n}{\argmin} \left( w_{\sigma} \right) = \underset{\sigma \in S_n}{\argmax} \frac{e^{-w_{\sigma}}}{\sum_{\sigma' \in S_n} e^{-w_{\sigma'}}},
	\end{equation*}
\end{frame}

\begin{frame}
	\frametitle{Metropolis-Hastings algorithm}
	\begin{equation*}
		\frac{\pi_j \bfQ_{j,i}}{\pi_i \bfQ_{i,j}} = e^{-(w_j-w_i)} \cdot \frac{\bfQ_{j,i}}{\bfQ_{i,j}}.
	\end{equation*}
	\begin{equation*}
		\log \left( \frac{\pi_j \bfQ_{j,i}}{\pi_i \bfQ_{i,j}} \right) = -(w_j-w_i) + \log(\bfQ_{j,i}) - \log(\bfQ_{i,j}).
	\end{equation*}
\end{frame}

\begin{frame}
	\frametitle{Candidates}
	\begin{definition}
		A neighbour $\sigma'$ of a permutation $\sigma$ is a permutation, that for some $k, l$ it satisfies $\sigma'(k) = \sigma(l)$, $\sigma'(l) = \sigma(k)$ and $\sigma'(i) = \sigma(i)$ for the rest of indices.
	\end{definition}
	These neighbours are the original tour with two swaped indices. This let  us consider a smaller space -- there are $\binom{n}{2} = \frac{n(n-1)}{2} \approx n^2$ neighbours if the number of vertices is $n$.
\end{frame}

\begin{frame}
	\frametitle{Random candidates (RN)}
	Sample neighbours uniformly. It is equivalent to choosing random indices to swap
\end{frame}

\begin{frame}
	\frametitle{Random candidates (RN)}
	When given a tour $\sigma$ and its neighbour $\sigma'$ they differ only on those edges where swap is happening, let us say $k,l$. So for this situation we have tours:
	\begin{align*}
		\sigma = (&\ldots, \sigma(k-1), \sigma(k), \sigma(k+1), \ldots, \\
		&\ldots, \sigma(l-1), \sigma(l), \sigma(l+1), \ldots) \\
		\sigma' = (&\ldots, \sigma(k-1), \sigma(l), \sigma(k+1), \ldots, \\
		&\ldots, \sigma(l-1), \sigma(k), \sigma(l+1), \ldots)
	\end{align*}
	We need to remove weights $w_{\sigma(k-1), \sigma(k)}$, $w_{\sigma(k), \sigma(k+1)}$, $w_{\sigma(l-1), \sigma(l)}$, $w_{\sigma(l), \sigma(l+1)}$ and add $w_{\sigma(k-1), \sigma(l)}, w_{\sigma(l), \sigma(k+1)}, w_{\sigma(l-1), \sigma(k)}, w_{\sigma(k), \sigma(l+1)}$.
\end{frame}

\begin{frame}
	\frametitle{Random candidates (RN)}
	\begin{algorithm}[H]
	\tiny
	\caption{Random neighbours algorithm}\label{alg:rnd_neigh}
	\begin{algorithmic}[1]
		\State{Choose a tour $\sigma \in S_n$.}
		\State{$X_0 \gets \sigma$}
		\State{Compute weight $w_\sigma$.}
		\For{$i = 0,1, \ldots$}
		\State{Sample $k, l \sim Unif\left\{1, 2, \ldots, n\right\}$ without replacement.}
		\State{Sample $U \sim Unif(0,1)$.}
		\State{$w_{\sigma'} \gets w_\sigma - (w_{\sigma(k-1) + \sigma(k)} + w_{\sigma(k) + \sigma(k+1)}, w_{\sigma(l-1) + \sigma(l)} + w_{\sigma(l), \sigma(l+1)}) $}
		\State{$w_{\sigma'} \gets w_\sigma' + (w_{\sigma(k-1) + \sigma(l)} + w_{\sigma(l) + \sigma(k+1)} + w_{\sigma(l-1) + \sigma(k)} + w_{\sigma(k) + \sigma(l+1)})$}
		\If{$\log(U) \leq \min \left(0, -(w_{\sigma'} - w_\sigma) \right)$}
		\State $X_{i+1} \gets X_i$
		\State $X_{i+1}(k), X_{i+1}(l) \gets X_{i+1}(l), X_{i+1}(k)$
		\State{$w_{\sigma} \gets w_{\sigma'}$}
		\Else
		\State $X_{i+1} \gets X_i$
		\EndIf
		\EndFor
	\end{algorithmic}
\end{algorithm}
\end{frame}

\begin{frame}
	\frametitle{Locally-informed proposals (LIP)}
	The idea is to balance the increase in the probability of neighbour with decrease of reverse probability, such that it will be easy to compute.
	\begin{equation*}
		\bfQ_{i,j} \propto e^\frac{-(w_{j} - w_i)}{\tau}.
	\end{equation*}
	The distribution is chosen in such a way, so that we can easily group up the terms in acceptance criterion:
	\begin{equation*}
		\frac{\pi_j \bfQ_{j,i}}{\pi_i \bfQ_{i,j}} = e^{-(w_j-w_i)} \cdot \frac{e^\frac{-(w_{i} - w_j)}{\tau}}{e^\frac{-(w_{j} - w_i)}{\tau}} \cdot \frac{C_j}{C_i} = e^{\left( -(w_j-w_i) \left(1 - \frac{2}{\tau}\right) \right)} \cdot \frac{C_j}{C_i},
	\end{equation*}
\end{frame}

\begin{frame}
	\frametitle{Locally-informed proposals (LIP)}
	Choose tour $\sigma$ and its neighbour $\sigma'$ that is connected with swapping indices $k$ and $l$.
	\begin{align*}
		\sigma = (&\ldots, \sigma(k-1), \sigma(k), \sigma(k+1), \ldots, \\
		&\ldots, \sigma(l-1), \sigma(l), \sigma(l+1), \ldots) \\
		\sigma' = (&\ldots, \sigma(k-1), \sigma(l), \sigma(k+1), \ldots, \\
		 &\ldots, \sigma(l-1), \sigma(k), \sigma(l+1), \ldots)
	\end{align*}
\end{frame}

\begin{frame}
	\frametitle{Locally-informed proposals (LIP)}
	The neighbours $\sigma_{r,s}$, $\sigma'_{r,s}$ of $\sigma$ and $\sigma'$ respectively look like: 
	\begin{align*}
	\sigma_{r,s} = &(\ldots, \sigma(r-1), \sigma(s), \sigma(r+1), \ldots, \\
	&\ldots, \sigma(s-1), \sigma(r),\sigma(s+1), \ldots, \\
	&\ldots, \sigma(k-1), \sigma(k), \sigma(k+1), \ldots, \\
	& \ldots, \sigma(l-1), \sigma(l), \sigma(l+1), \ldots) \\
	\sigma'_{r,s} = &(\ldots, \sigma(r-1), \sigma(s), \sigma(r+1), \ldots \\
	& \ldots, \sigma(s-1), \sigma(r), \sigma(s+1), \ldots, \\
	&\ldots \sigma(k-1), \sigma(l), \sigma(k+1), \ldots, \\
	& \ldots, \sigma(l-1), \sigma(k), \sigma(l+1), \ldots)
	\end{align*}
\end{frame}

\begin{frame}
	\frametitle{Example}
	Let us set $k=2$ and $l=7$, then the set of indices to consider is $\left\{ 1,2,3,6,7,8 \right\}$ and the permutations:
	\begin{align*}
	\sigma &= (1, 2, 3, 4, 5, 6, 7, 8, 9, \ldots) \\
	\sigma' &= (1, 7, 3, 4, 5, 6, 2, 8, 9, \ldots).
	\end{align*}
\end{frame}

\begin{frame}
	\frametitle{Example}
	\begin{table}[!htb]
		\footnotesize
		\centering
		\begin{tabular}{lllllllllll}
			& 2                             & 3                             & 4                            & 5                             & 6                             & 7                             & 8                             & 9                             & 10                                &  \\
			1 & \cellcolor[HTML]{9B9B9B}(1,2) & \cellcolor[HTML]{9B9B9B}(1,3) & \cellcolor[HTML]{C0C0C0}(1,4) & \cellcolor[HTML]{C0C0C0}(1,5) & \cellcolor[HTML]{9B9B9B}(1,6) & \cellcolor[HTML]{9B9B9B}(1,7) & \cellcolor[HTML]{9B9B9B}(1,8) & \cellcolor[HTML]{C0C0C0}(1,9) & \cellcolor[HTML]{C0C0C0}$\cdots$ &  \\
			2 &                               & \cellcolor[HTML]{9B9B9B}(2,3) & \cellcolor[HTML]{C0C0C0}(2,4) & \cellcolor[HTML]{C0C0C0}(2,5) & \cellcolor[HTML]{9B9B9B}(2,6) & \cellcolor[HTML]{9B9B9B}(2,7) & \cellcolor[HTML]{9B9B9B}(2,8) & \cellcolor[HTML]{C0C0C0}(2,9) & \cellcolor[HTML]{C0C0C0}$\cdots$ &  \\
			3 &                               &                               & \cellcolor[HTML]{C0C0C0}(3,4) & \cellcolor[HTML]{C0C0C0}(3,5) & \cellcolor[HTML]{9B9B9B}(3,6) & \cellcolor[HTML]{9B9B9B}(3,7) & \cellcolor[HTML]{9B9B9B}(3,8) & \cellcolor[HTML]{C0C0C0}(3,9) & \cellcolor[HTML]{C0C0C0}$\cdots$ &  \\
			4 &                               &                               &                               & (4,5)                         & \cellcolor[HTML]{C0C0C0}(4,6) & \cellcolor[HTML]{C0C0C0}(4,7) & \cellcolor[HTML]{C0C0C0}(4,8) & (4,9)                         & $\cdots$                         &  \\
			5 &                               &                               &                               &                               & \cellcolor[HTML]{C0C0C0}(5,6) & \cellcolor[HTML]{C0C0C0}(5,7) & \cellcolor[HTML]{C0C0C0}(5,8) & (5,9)                         & $\cdots$                         &  \\
			6 &                               &                               &                               &                               &                               & \cellcolor[HTML]{9B9B9B}(6,7) & \cellcolor[HTML]{9B9B9B}(6,8) & \cellcolor[HTML]{C0C0C0}(6,9) & \cellcolor[HTML]{C0C0C0}$\cdots$ &  \\
			7 &                               &                               &                               &                               &                               &                               & \cellcolor[HTML]{9B9B9B}(7,8) & \cellcolor[HTML]{C0C0C0}(7,9) & \cellcolor[HTML]{C0C0C0}$\cdots$ &  \\
			8 &                               &                               &                               &                               &                               &                               &                               & \cellcolor[HTML]{C0C0C0}(8,9) & \cellcolor[HTML]{C0C0C0}$\cdots$ &  \\
			9 &                               &                               &                               &                               &                               &                               &                               &                               & $\ddots$                         &  \\
			&                               &                               &                               &                               &                               &                               &                               &                               &                                  & 
		\end{tabular}
		\caption{Neighbour representation.}
	\end{table}
\end{frame}

\begin{frame}
	\frametitle{Example}
	Let us set $r=3$ and $s=9$, which means that we are looking for weight differences of neighbours obtained by swapping $3$ and $9$. So the neighbours have form:
	\begin{align*}
		\sigma_{3,9} &= (1, 2, 9, 4, 5, 6, 7, 8, 3, \ldots) \\
		\sigma'_{3,9} &= (1, 7, 9, 4, 5, 6, 2, 8, 3, \ldots).
	\end{align*}
	\begin{equation*}
	w_{\sigma_{3,9}} - w_{\sigma'_{3,9}} = (w_{1,2} + w_{2,9} + w_{6,7} + w_{7,8}) - (w_{1,7} + w_{7,9} + w_{6,2} + w_{2,8}).
	\end{equation*}
	We can generalize that equation for any $p \notin \left\{ 1,2,3,6,7,8 \right\}$:
	\begin{equation*}
	w_{\sigma_{3,p}} - w_{\sigma'_{3,p}} = f_3(p) = C + w_{2,p} - w_{7,p}.
	\end{equation*}
\end{frame}

\begin{frame}
	\frametitle{Locally-informed proposals (LIP)}
	\begin{algorithm}[!htb]
	\caption{Locally-informed proposals algorithm}\label{alg:loc_neigh}
	\begin{algorithmic}[1]
		\State{Choose a tour $\sigma \in S_n$.}
		\State{$X_0 \gets \sigma$}
		\State{Compute weight $w_\sigma$.}
		\State{Compute all neighbour weight differences $\bfd_\sigma$.}
		\State{$s(\bfd_\sigma) = \mathrm{softmax}(\bfd_\sigma)$}
		
		\For{$i = 1,2, \ldots$}
			\State{Sample $\sigma' \sim s(\bfd_\sigma)$.}
			\State{Find $k, l$ connected with swapping.}
			
			\State{$w_{\sigma'} \gets w_\sigma + \bfd_\sigma[(k,l)]$}
			\State{$\bfd_{\sigma'} \gets \bfd_\sigma$}
			
			\State{$\bfd_{\sigma'} \gets$ \textit{update\_differences}$(\bfd_{\sigma'})$}
			
			\State{$s(\bfd_{\sigma'}) = \mathrm{softmax}(\bfd_{\sigma'})$}
			\State{Sample $U \sim Unif(0,1)$.}
			\If{$\log(U) \leq \min \left(0, -(w_{\sigma'} - w_\sigma) + \log(s(\bfd_{\sigma'})[(k,l)]) - \log(s(\bfd_\sigma)[(k,l)]) \right)$}
				\State $X_{i+1}(k), X_{i+1}(l) \gets X_{i+1}(l), X_{i+1}(k)$
				\State{$w_{\sigma} \gets w_{\sigma'}$}
				\State $\bfd_{\sigma} \gets \bfd_{\sigma'}$
			\Else
				\State $X_{i+1} \gets X_i$
			\EndIf
		\EndFor
	\end{algorithmic}
\end{algorithm}
		
The main algorithm uses a sub-algorithm \ref{alg:update_diff} which updates weights in swaps that we need to consider separately. It uses function \textit{get\_difference}, which is just getting appropriate weights to remove and add. 
\begin{algorithm}[!htb]
	\caption{update\_differences}\label{alg:update_diff}
	\begin{algorithmic}[1]
		\Require $\bfd_{\sigma'}$
		\Ensure $\bfd_{\sigma'}$
		\For{$r = 1, 2, \ldots, n$}
			\For{$s = r+1, \ldots, n$}
				\If{$r \; \mathrm{or} \; s \; \mathrm{in} \left\{ (k-1), k, (k+1), (l-1), l, (l+1) \right\}$}
					\State{$\bfd_{\sigma'}[(m,n)] \gets$ \textit{get\_difference}($\bfd_{\sigma'}[(r,s)]$)}
				\EndIf
			\EndFor
		\EndFor
	\end{algorithmic}
\end{algorithm}

\begin{algorithm}[!htb]
	\caption{update\_differences\_alt}\label{alg:update_diff_alt}
	\begin{algorithmic}[1]
		\Require $\bfd_{\sigma'}$
		\Ensure $\bfd_{\sigma'}$
		\For{$r = k-1, k, k+1, l-1, l, l+1$}
			\For{$s = r+1, \ldots, n$}
				\State{$\bfd_{\sigma'}[(m,n)] \gets$ \textit{get\_difference}($\bfd_{\sigma'}[(r,s)]$)}
			\EndFor
		\EndFor
		
		\For{$r = k+2, k+3, \ldots, l-2$}
			\For{$s = l-1, l, l+1$}
				\State{$\bfd_{\sigma'}[(m,n)] \gets$ \textit{get\_difference}($\bfd_{\sigma'}[(r,s)]$)}
			\EndFor
		\EndFor
	\end{algorithmic}
\end{algorithm}
Alternate algorithm explanation: ??
\end{frame}

\begin{frame}
	\frametitle{Locally-informed proposals (LIP)}
	\begin{algorithm}[H]
	\tiny
	\caption{update\_differences}
	\begin{algorithmic}[1]
		\Require $\bfd_{\sigma'}$
		\Ensure $\bfd_{\sigma'}$
		\For{$r = k-1, k, k+1, l-1, l, l+1$}
		\For{$s = r+1, \ldots, n$}
		\State{$\bfd_{\sigma'}[(r,s)] \gets$ \textit{get\_difference}($\bfd_{\sigma'}[(r,s)]$)}
		\EndFor
		\EndFor
		
		\For{$s = k-1, k, k+1, l-1, l, l+1$}
		\For{$r = 1, 2, \ldots l$}
		\State{$\bfd_{\sigma'}[(r,s)] \gets$ \textit{get\_difference}($\bfd_{\sigma'}[(r,s)]$)}
		\EndFor
		\EndFor
	\end{algorithmic}
\end{algorithm}
\end{frame}

\begin{frame}
	\frametitle{Simulated annealing}
	The idea is to describe a probability of state using a cooling parameter $t_k$ such that it reminds the cooling of a metal and may change with each step:
	\begin{equation*}
	\pi_i = \frac{e^{\frac{-E_i}{t_k}}}{C},
	\end{equation*}
	where $C$ is normalizing constant.The quotient of probabilities then is:
	\begin{equation*}
	\frac{\pi_j}{\pi_i} = e^{\frac{E_i - E_j}{t_k}}
	\end{equation*}
\end{frame}