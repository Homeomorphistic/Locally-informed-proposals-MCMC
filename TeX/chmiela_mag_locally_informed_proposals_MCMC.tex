\documentclass[a4paper, 12pt]{article}
\usepackage[utf8]{inputenc}
\usepackage[T1]{fontenc}
%\usepackage[polish]{babel}
\usepackage[top=2.5cm, bottom=2.5cm, left=2.5cm, right=2.5cm]{geometry}
\usepackage{amsmath,amsthm}
\usepackage{hyperref}
\usepackage{enumitem}
\usepackage{subcaption}
\usepackage{subcaption}
\usepackage{float}

\usepackage{listings}
\lstset{breaklines=true, numbers=left, stepnumber=2, language=Python}

\numberwithin{equation}{subsection}
\counterwithin{table}{subsection}
\counterwithin{figure}{subsection}

\begin{document}
	\thispagestyle{empty}
	\begin{center}
		\textbf{\large Uniwersytet Wroc\l{}awski\\
			Wydzia\l{} Matematyki i Informatyki\\
			Instytut Matematyczny}\\
		\textit{\large specjalno\'{s}\'{c}: Analiza danych}\\
		\vspace{4cm}
		\textbf{\textit{\large Bartosz Chmiela}\\
			\vspace{0.5cm}
			{\Large Locally-informed proposals in Metropolis-Hastings algorithm with applications}}\\
	\end{center}
	\vspace{3cm}
	{\large \hspace*{6.5cm}Praca magisterska\\
		\hspace*{6.5cm}napisana pod kierunkiem\\
		\hspace*{6.5cm}dr hab. Paw\l{}a Lorka }\\
	\vfill
	\begin{center}
		{\large Wroc\l{}aw 2022}\\
	\end{center}
	
	\newpage
	\vspace*{\fill}
	\begin{abstract}
		The Markov Chain Monte Carlo methods (abbrv. MCMC) are a family of algorithms used for sampling from a given probability distribution. They prove very effective when the state space is large. This fact can be used to solve many hard deterministic problems -- one of them being \textit{traveling salesmen problem}. It will be used in this thesis to test a new approach of \textit{locally-informed propolsals} as a modification of well known \textit{Metropolis-Hastings} algorithm. In this thesis we will present the implementation of modified algorithm, experiments based on it, results and a comparison of to previous MCMC methods.
		
		\rule{0.8\textwidth}{0.4pt}
		
		Metody próbkowania Monte Carlo łańcuchami Markowa są rodziną algorytmow używanych do próbkowania z danego rozkładu prawdopodobieństwa. Okazują się efektywne zwłaszcza gdy przestrzeń stańw jest wielka. Ten fakt może być wykorzystany przy rozwiązywaniu wielu deterministycznych problemów -- jednym z nich jest \textit{problem komiwojażera}. Zostanie on użyty w tej pracy do przetestowania nowego podejścia \textit{lokalnie poinformowawnego?}, jako modyfikacji dobrze znanego algorytmu \textit{Metropolisa-Hastinigsa}. W tej pracy zaprezentujemy implementacje zmodyfikowanego algorytmu, eksperymentów bezujących na nim, wyników oraz porównania z poprzednimi metodami próbkowania Monte Carlo.
	\end{abstract}
	\vspace*{\fill}
	\clearpage
	
	\tableofcontents
	\listoftables
	\listoffigures
	\clearpage
	
	\section{Introduction}
		The Markov Chain Monte Carlo methods (abbrv. MCMC) are a family of algorithms used for sampling from a given probability distribution. At first they do not seem useful for solving practical deterministic problems, but with some tweaks they can become a powerful tool. It happens especially when space of possible solutions is enormous and computing becomes infeasible for machines. These offer a shortcut for obtaining ``close enough'' answers.
		
		At their core, MCMC methods generate a Markov Chain (abbrv. MC) with a defined distribution and sample using it. The convergence of the chain is assured by ergodic theorems. The most known of them is \textit{Metropolis-Hastings} algorithm, which constructs a MC using another set of distributions, maybe simpler ones.
		
		In this thesis we work on \textit{locally-informed proposals}, which involve determining \textit{local} distribution -- which comes down to finding transition probabilities of the state. They are a bit more complex and computationally heavy, but offer better results with less iterations. 
		
		To test this method we will need a deterministic problem which quickly becomes infeasible for machines to compute -- one of them is a well-known traveling salesman problem. The testing is carried out using its benchmark training set \textit{tsplib95} and implemen-tation is provided in \textit{Python3}.
	
	\section{Markov Chains}
	
	\section{Markov Chain Monte Carlo methods}
	
	\section{Traveling salesman problem}
	
	\section{Decoding encrypted text}
	
	\section{Code description?}
	
	\section{Conclusions}
	
	\clearpage
	\addcontentsline{toc}{section}{References}
	\nocite{*}
	\bibliographystyle{abbrv}
	\bibliography{references}
	
	
	
	\appendix
	\clearpage
	\section{Source code} \label{apsec:code}
		%\lstinputlisting[firstline=13, lastline=192]{genomeTesting.R}
	\clearpage
	
	
\end{document}